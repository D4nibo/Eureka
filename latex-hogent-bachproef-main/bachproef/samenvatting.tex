%%=============================================================================
%% Samenvatting
%%=============================================================================

% TODO: De "abstract" of samenvatting is een kernachtige (~ 1 blz. voor een
% thesis) synthese van het document.
%
% Een goede abstract biedt een kernachtig antwoord op volgende vragen:
%
% 1. Waarover gaat de bachelorproef?
% 2. Waarom heb je er over geschreven?
% 3. Hoe heb je het onderzoek uitgevoerd?
% 4. Wat waren de resultaten? Wat blijkt uit je onderzoek?
% 5. Wat betekenen je resultaten? Wat is de relevantie voor het werkveld?
%
% Daarom bestaat een abstract uit volgende componenten:
%
% - inleiding + kaderen thema
% - probleemstelling
% - (centrale) onderzoeksvraag
% - onderzoeksdoelstelling
% - methodologie
% - resultaten (beperk tot de belangrijkste, relevant voor de onderzoeksvraag)
% - conclusies, aanbevelingen, beperkingen
%
% LET OP! Een samenvatting is GEEN voorwoord!

%%---------- Nederlandse samenvatting -----------------------------------------
%
% TODO: Als je je bachelorproef in het Engels schrijft, moet je eerst een
% Nederlandse samenvatting invoegen. Haal daarvoor onderstaande code uit
% commentaar.
% Wie zijn bachelorproef in het Nederlands schrijft, kan dit negeren, de inhoud
% wordt niet in het document ingevoegd.

\IfLanguageName{english}{%
\selectlanguage{dutch}
\chapter*{Samenvatting}
\lipsum[1-4]
\selectlanguage{english}
}{}

%%---------- Samenvatting -----------------------------------------------------
% De samenvatting in de hoofdtaal van het document

\chapter*{\IfLanguageName{dutch}{Samenvatting}{Abstract}}

Tijdens hun opleiding aan HOGENT worden studenten geconfronteerd met een grote hoeveelheid informatie over vakken, lesmodaliteiten, deadlines en softwarevereisten. Ondanks de beschikbaarheid van studiemateriaal en andere informatiebronnen blijven er vaak vragen onbeantwoord. Studenten wenden zich eerst tot medestudenten op sociale media om antwoorden te vinden, maar dit kan resulteren tot foutieve of tegenstrijdige informatie. Wanneer ze alsnog geen duidelijk antwoord krijgen, nemen ze contact op met de docent. Dit leidt tot een verhoogde werkdruk en een inefficiënt informatie-uitwisselingsproces.

Deze bachelorproef onderzoekt de haalbaarheid en toepasbaarheid van Eureka, een virtuele assistent die gebruikmaakt van Large Language Models (LLM’s) om studenten en docenten te ondersteunen bij vakgerelateerde vragen. Eureka moet niet alleen correcte en relevante antwoorden genereren, maar ook bronnen vermelden en hallucinaties vermijden. Bovendien moet het systeem robuust zijn tegen ongewenste interacties en eenvoudig uitbreidbaar, zodat ook docenten zonder expertise in LLM-technologie het kunnen gebruiken.

Aangezien het ontwikkelen of volledig fine-tunen van een LLM te veel middelen vereist, wordt gekozen voor Retrieval-Augmented Generation (RAG). Deze techniek combineert een LLM met een externe databank waarin cursusmateriaal wordt opgeslagen. Wanneer een student een vraag stelt, haalt Eureka de relevante informatie op uit de databank en gebruikt deze als context om een accuraat antwoord te genereren. Hierdoor kan het systeem goed functioneren binnen de dynamische onderwijsomgeving, waarin vakinhoud en leermaterialen regelmatig veranderen.

Om Eureka te evalueren, wordt een Proof-of-Concept (PoC) ontwikkeld en getest met vakken gegeven door dhr. Van Vreckem. Dit PoC wordt lokaal opgezet en wordt gezocht naar de ideale LLM, embedder en chunking-techniek combinatie. De PoC wordt geëvalueerd door de gegenereerde antwoorden te laten beoordelen door zowel een vertegenwoordiger van de docenten als een vertegenwoordiger van de studenten. De evaluatiecriteria zijn relevantie, correctheid, bronvermelding en afwezigheid van hallucinaties.

Indien succesvol, wordt Eureka uitgerold op een server en toegankelijk gemaakt via een REST API. Dit onderzoek toont aan dat een LLM-gebaseerde chatbot in combinatie met RAG een efficiënte oplossing kan bieden voor de informatievoorziening binnen HOGENT, waarbij studenten direct toegang krijgen tot betrouwbare antwoorden en de werkdruk van docenten wordt verminderd.