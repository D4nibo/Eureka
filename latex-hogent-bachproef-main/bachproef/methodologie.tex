%%=============================================================================
%% Methodologie
%%=============================================================================

\chapter{\IfLanguageName{dutch}{Methodologie}{Methodology}}%
\label{ch:methodologie}

%% TODO: In dit hoofstuk geef je een korte toelichting over hoe je te werk bent
%% gegaan. Verdeel je onderzoek in grote fasen, en licht in elke fase toe wat
%% de doelstelling was, welke deliverables daar uit gekomen zijn, en welke
%% onderzoeksmethoden je daarbij toegepast hebt. Verantwoord waarom je
%% op deze manier te werk gegaan bent.
%% 
%% Voorbeelden van zulke fasen zijn: literatuurstudie, opstellen van een
%% requirements-analyse, opstellen long-list (bij vergelijkende studie),
%% selectie van geschikte tools (bij vergelijkende studie, "short-list"),
%% opzetten testopstelling/PoC, uitvoeren testen en verzamelen
%% van resultaten, analyse van resultaten, ...
%%
%% !!!!! LET OP !!!!!
%%
%% Het is uitdrukkelijk NIET de bedoeling dat je het grootste deel van de corpus
%% van je bachelorproef in dit hoofstuk verwerkt! Dit hoofdstuk is eerder een
%% kort overzicht van je plan van aanpak.
%%
%% Maak voor elke fase (behalve het literatuuronderzoek) een NIEUW HOOFDSTUK aan
%% en geef het een gepaste titel.
 
\section{\IfLanguageName{dutch}{Requirementsanalyse}{Requirement-analysis}}%
\label{sec:requirement-analyse-meth}

De eerste stap die werd gezet, was het uitvoeren van een requirementsanalyse. Het doel van deze analyse was het in kaart brengen van de verschillende stakeholders en hun behoeften. Dit was een cruciale stap, aangezien het de stakeholders zijn die uiteindelijk baat hebben bij deze bachelorproef. De bachelorproef moet dan ook aansluiten bij hun behoeften om als succesvol te worden beschouwd. Het resultaat van de requirementsanalyse diende als maatstaf om het project Eureka aan af te toetsen.

De requirementsanalyse werd in twee fasen uitgevoerd: het opstellen van persona’s en het formuleren van de vereisten zelf. Persona’s maakten het mogelijk om een beeld te krijgen van de stakeholders en wat zij belangrijk vonden. Op basis van deze persona’s werden vervolgens de requirements opgesteld.

\section{\IfLanguageName{dutch}{Verkenning van de onderdelen van de RAG-pijplijn}{Exploration of the components of the RAG pipeline}}%
\label{sec:verkenning-onderdelen}

Zoals besproken in sectie~\ref{subsec:rag}, bestaat de RAG-pijplijn uit verschillende componenten, elk met hun eigen eigenschappen en uitdagingen. Om deze beter te begrijpen, werd elk onderdeel afzonderlijk onderzocht aan de hand van kleinschalige experimenten. Met andere woorden, de experimenten hadden tot doel technische inzichten te verwerven die later benut konden worden bij het opstellen van de PoC. Deze experimenten werden opgezet in Jupyter Notebooks. 

De structuur van de experimenten was als volgt:

\begin{enumerate}
    \item \textbf{Ontdekking van modellen}: twee soorten modellen werden onderzocht, namelijk het \acrshort{LLM} dat de interacties genereert en het model, dat via de \emph{Moderation API} van OpenAI beschikbaar wordt gesteld, dat ongepaste interacties moet uitfilteren.
    \item \textbf{Tekstmanipulatie}: in deze experimenten werd onderzocht hoe PDF-bestanden konden worden omgezet en hoe verschillende chunkingstrategieën konden worden toegepast.
    \item \textbf{Vectorisatie}: hier werd kennisgemaakt met een vectordatabank -- Chroma -- en het opslaan van chunks in deze databank.
\end{enumerate}

\section{\IfLanguageName{dutch}{Opzetten van de PoC}{PoC setup}}%
\label{sec:poc-opzetting}

Met de verworven inzichten uit de technische experimenten werd de PoC opgezet. Deze werd geïmplementeerd in de programmeertaal Python, met gebruik van \emph{Langchain} waar van toepassing. Langchain is een framework dat de integratie van \acrfull{LLM} in applicaties vergemakkelijkt.

Er werd gebruikgemaakt van het \emph{Strategy design pattern}. Dit patroon maakt het mogelijk om de strategie van verschillende componenten eenvoudig te wijzigen en de pijplijn op een flexibele manier uit te breiden met nieuwe strategieën.

\section{\IfLanguageName{dutch}{Evaluatie van de PoC}{PoC evaluation}}%
\label{sec:poc-opzetting}

In deze stap werd de PoC geëvalueerd op zijn geschiktheid voor de stakeholders. Hiervoor werd een door ons ontwikkeld evaluatieraamwerk toegepast.

Voor elke instelling waarin Eureka werd getest, werd per vak een reeks vragen opgesteld. Deze vragen konden betrekking hebben op de modaliteiten (zoals deadlines, doelstellingen, organisatie, enzovoort), op de inhoud van het vak, of geen directe relatie hebben met de vakinhoud.

De gegenereerde antwoorden werden vervolgens beoordeeld aan de hand van specifieke criteria. Voor elk criterium werd een score toegekend op een schaal van 1 tot 5, waarbij 1 staat voor \emph{zeer onvoldoende} en 5 voor \emph{uitstekend}.